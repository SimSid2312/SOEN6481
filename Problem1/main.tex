
\documentclass{article}
\usepackage[utf8]{inputenc}
\usepackage{amsmath}

\title{Problem 1}
\date{January 2019}

\begin{document}
\noindent
\large\textbf{Problem 1} \hfill \textbf{Simran Sidhu} \\
\normalsize SOEN6481 \hfill \textbf{40011611} \\
\section{Give a brief description, not exceeding one page, of your number, including the
characteristics that make it unique.}
\subsection*{Function :  Natural logarithm of 2 i.e. $ln_e{2}$}
Definitions :\\
Irrational Numbers - are the numbers that cannot be represented as ratio or a fraction.\\\\
Natural Logarithm -  The natural logarithm of a number x is nothing but log to the base e of x. Here  e has a approximate value of 2.718.\\
Natural logarithm is computing the time taken to reach the desired growth.\\
$log_e{x}$ can be written as ln x\\
 ln is called the natural log.\\\\
Natural Logarithm of 2 - The project is based on the natural logrithm of 2 ie. $ln_e{2}$ . \\
The value of $ln_e{2} \approx 0.69314718056$ and it is an irrational number i.e cannot be expressed in fractional form.

The proof of $ln_e{2}$ being irrational goes something like :\\
Let suppose, $ln_e{2}$ is rational i.e. there exist a x,y integers  $>0$ and they can represent the natural log of 2.\\
Therefore it can be said :\\ 
$ln_e{2}=x/y$\\
Applying exponential to both LHS and RHS , we get:\\
$e^{ln_e{2}}=e^{x / y}$\\
$2=e^{x / y}$\\
$2^y=e^{x}$\\

Since we know e is a transcendental number and from the theorm mentioned the famous book - "Proofs from the book" [1],Page 45, $e^r$ , where r is rational number not equal 0 , is irrational we can say that $ln_e{2}$ is also an irrational number i.e. cannot be denoted as ratio of two integers with value $>0$. The understanding of the proof was gathered from the website [2] - concept explained by Richard Morris, Maths tutor, doctorate in mathematics/computer science.

\subsection*{Application of natural logarithm of 2}
The uniqueness of this number has been noticed in below concepts:

1. Half-life : Natural Logarithm of 2 plays a significant role in computing half life of a substance i.e computing the time taken by a substance to reduce to half of it initial value.This is concept is used in nuclear physics and biology.\\\\
2. Finance - The Rule 72 : Natural Logarithm of 2 is used in finance sector as a way to quickly compute annually computed interest and continously compounded interest.  i.e. when we have to find the time taken (in years) to double the principle at a given interest rate, we have to divide 72 by interest rate(given). And this number 72 is calculated using natural logarithm of 2.

\subsection*{Reference}
1. Aigner, Martin, and Günter M. Ziegler. Proofs from THE BOOK. Fourth ed.\\
2. “How Do I Prove ln2 Is Irrational?” Quora, www.quora.com/How-do-I-prove-ln2-is-irrational.



\end{document}



 