\documentclass{article}
\usepackage[utf8]{inputenc}
\usepackage{amsmath}

\title{Problem 2}
\date{January 2019}

\begin{document}

\noindent
\large\textbf{Problem 2} \hfill \textbf{Simran Sidhu} \\
\normalsize SOEN6481 \hfill \textbf{40011611} \\
\section{Introduction about the interviewee}
The interviewee of this project is : Ms. Nileesha Fernando who is currently per suing her graduation in Master of Software Engineering and has completed her bachelors in computer science. Apart from this she is also working part time as Full Stack PHP intern at a software company in Montreal.
My reasons for choosing her are as follows :\\
1. Completed a course in System Requirement Specification at Concordia with a grade - A+  under professor - Abdelwahab Elnaka in her masters.\\ 
2. She coded a calculator using JavaScript,HTML,CSS as a side project.\\
3. She took various other courses involving complex mathematical computation both in masters and bachelors.
\section{Interview}
Before interviewing her , I briefly explained what my project is about and gave an overview of what knowledge i have gathered so far about the number assigned to me i.e. natural logarithm of 2. Then we discussed about the irrationality of natural logarithm of 2. 

Q1 - Have you used natural logarithms in any of previous work or as a school project ? \\\\
Ans - I used natural logarithms in my undergraduate course like - physics,machine learning ,artificial intelligence,statistics.Also i took 
Algorithm Design Techniques in my masters. Since all these course involves computing complex equation hence i have used natural logarithm many times. Not only this , in some hard to solve problems the use of natural logarithm made it easier for me to compute them.\\
\\
Q2 - For the natural logarithm of 2  did you use it and was it rarely used or frequently used ?

Ans - I have used natural logarithms a lot in courses like Algorithm Design, however as far as i can remember , i didn't really used logarithm of 2. But  I did used natural logarithm of a number specially in topics like finding the complexity i.e  Big-O notation problems.Actually, using natural logrithm was fun as all the properties of logrithm i.e Multiplication,division,power etc are also applicable to natural logrithm so it was interesting to compute log or ln involving equations.\\
\\
Q3 - Could you illustrate an example for me which can demonstrate the fact that using a natural logarithm is helpful?\\
Ans - I will explain it using an example , we used to solve question like - 
simplify the express:\\

$ln (e^{-3})$ \\
$\implies ln_e(e^{-3})$ = $-3$
We were solving such question but if i have to solve a number that has  
$ln_e(2)$  here we have a number 2 inside the bracket i would use a calculator for solving an expression involving it , as manually i do not think i can further simplify it.\\
\\
Q4 - Do you prefer solving it using a calculator or manually ?\\
Ans - I would be inclined towards using a calculator to quickly compute the natural logarithm of number as we just have to press the button Ln and then enter the number after it.\\
\\
Q5 - Any challenges you faced while using this number i.e. Natural logrithm of 2 with or without a calculator?\\
Ans- I would use a calculator to solve problems using an expression Ln(2) as its  natural log of base e is of 2 - a number and not an expression that has exponential form.\\
\\
Q6 - Generic question about the scientific calculator , can you share your past experience of using a scientific calculator ?\\
Ans- I have used the calculator for educational purposes. In my mathematics and physics classes, it was vital to use the scientific calculator during lab experiments to perform data analysis. I used the Casio S-V P.A.M calculator and sometimes I used online scientific calculators to perform more complex scientific equations. The main drawback with my current scientific calculator is that it cannot perform complex functionality in a simple manner.\\
\\
Q6 - Any feature you feel should be there in the calculator to make it easier for the user to perform complex mathematical equation easily?\\

Ans - This might not be related to the function you have been asked for your project but rather a feature that i think i dont have on my scientific calculator is computing the log base x of n , where x can be any number other than 10 and n is any number . \\
How i have to calculate it using a scientific calculator is : 
 $(log_{10} n ) / log_{10} b$ 
I know that Casio calculator do have the feature of compute the log of any base of a number but i feel they should make this feature a must for all the calculators.\\ 

Q7- When you use natural logarithm of a number do you round off the digits and if so how many decimal place do you round off the result ?\\ 
Ans - As far as my memory goes , i was always rounding off a natural logarithm of a number up to 3 decimal place like for instance in your function i know the 
$ln_{e} 2$ is an irrational number with a value 0.69314... .\\
So in this case i would just round off it to 0.6931 and use it in my equation to reduce it further.

\section{Analysis about Interview}
Interviewing Ms. Nileesha Frenando suggested that she has used the concept - natural logrithm in many significant courses of her major.However has not dealt with natural logrithm of 2 in particular.She mentioned that if she needs to solve an complex equation that has $ln_{e}(2)$ she would use value returned by a calculator (scientific calculator) for the function and the reason she mentiones is that as it has no exponential term rather has a number 2 which according to her leaves her with no means to further simplify it manually.To add to that she mentioned a must be present feature in all the calculator i.e. a single press button to compute the log of any base of a number.
\end{document}
