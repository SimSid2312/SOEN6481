\documentclass{article}
\usepackage[utf8]{inputenc}
\usepackage{amsmath}

\title{Problem 2}
\date{January 2019}

\begin{document}

\noindent
\large\textbf{Problem 2} \hfill \textbf{Simran Sidhu} \\
\normalsize SOEN6481 \hfill \textbf{40011611} \\
\section{Introduction about the interviewee}
The interviewee of this project is : Ms. Nileesha Fernando who is currently per suing her graduation in Master of Software Engineering and has completed her bachelors in computer science. Apart from this she is also working part time as Full Stack PHP intern at a software company in Montreal.
My reasons for choosing her are as follows :\\
1. Completed a course in System Requirement Specification at Concordia with a grade - A+  under professor - Abdelwahab Elnaka in her masters.\\ 
2. She coded a calculator using JavaScript,HTML,CSS as a side project.\\
3. She took various other courses involving complex mathematical computation both in masters and bachelors.
\section{Interview}
Q1 - Have you used natural logarithms in any of pervious work or as a school project ? \\\\
Ans - I used natural logarithms in my undergraduate course like - physics,machine learning ,artificial intelligence,statistics.Also i took 
Algorithm Design Techniques in my masters. Since all these course involves computing complex equation hence i have used natural logarithm many times. Not only this , in some hard to solve problems the use of natural logarithm made it easier for me to compute them.\\
\\
Q2 - For the natural logarithm of 2  did you use it and was it rarely used or frequently used ?

Ans - I would say in courses like Algorithm Design I used natural logarithm of a number frequently specially in topics like finding the complexity i.e  Big-O notation problems.Actually, using natural logrithm was fun as all the properties of logrithm i.e Multiplication,division,power etc are also applicable to natural logrithm so it was interesting to compute log or ln involving equations.\\
\\
Q3 - Could you illustrate for an example for me which can demonstrate the fact that using a natural logarithm is helpful?\\


Q4 -did you use it i.e. Do you prefer solving it using a calculator or manually ?\\
Q4 - Any challenges you faced while using this number with or without a calculator?\\
Q5 - Generic question about the calculator can you share your past experience of using a scientific calculator ?\\
Q6 - Any feature you feel should be there in the calculator to make it easier for the user to perform complex mathematical equation easily?\\
Q7- When you use natural logarithm of a number do you round off the digits and if so how many decimal place do you round off the result ? 


\section{Analysis about Interview}

\end{document}
