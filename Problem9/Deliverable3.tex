\documentclass[final,hyperref={pdfpagelabels=false}]{beamer}
\usepackage{grffile}
\mode<presentation>{\usetheme{I6pd2}}
\usepackage[english]{babel}
%\usepackage[latin1]{inputenc}
\usepackage{amsmath,amsthm, amssymb, latexsym}

\boldmath
\usepackage[orientation=portrait,size=a0,scale=1.4,debug]{beamerposter}
% change list indention level
% \setdefaultleftmargin{3em}{}{}{}{}{}


%\usepackage{snapshot} % will write a .dep file with all dependencies, allows for easy bundling

\usepackage{array,booktabs,tabularx}
\newcolumntype{Z}{>{\centering\arraybackslash}X} % centered tabularx columns
\newcommand{\pphantom}{\textcolor{ta3aluminium}} % phantom introduces a vertical space in p formatted table columns??!!

\listfiles

%%%%%%%%%%%%%%%%%%%%%%%%%%%%%%%%%%%%%%%%%%%%%%%%%%%%%%%%%%%%%%%%%%%%%%%%%%%%%%%%%%%%%%
\setauthorurl{GitHub Link: https://github.com/SimSid2312/SOEN6481.git}
\setauthoremail{simrantechm@gmail.com}

\title{\huge{Enternity:Numbers}}
\author{Simran Sidhu(40011611,Team-F)} 
\institute[ui]{SOEN6481:Software Systems Requirements Specification\\
Function : Natural Logarithm of 2 i.e $ln_{e}2$
}

%%%%%%%%%%%%%%%%%%%%%%%%%%%%%%%%%%%%%%%%%%%%%%%%%%%%%%%%%%%%%%%%%%%%%%%%%%%%%%%%%%%%%%
\newlength{\columnheight}
\setlength{\columnheight}{105cm}
%%%%%%%%%%%%%%%%%%%%%%%%%%%%%%%%%%%%%%%%%%%%%%%%%%%%%%%%%%%%%%%%%%%%%%%%%%%%%%%%%%%%%%
\begin{document}
\begin{frame}

 \begin{columns}
    % ---------------------------------------------------------%
    % Set up a column 
    \begin{column}{.49\textwidth}
      \begin{beamercolorbox}[center,wd=\textwidth]{postercolumn}
        \begin{minipage}[T]{.95\textwidth}  % tweaks the width, makes a new \textwidth
          \parbox[t][\columnheight]{\textwidth}{ % must be some better way to set the the height, width and textwidth simultaneously
            % Since all columns are the same length, it is all nice and tidy.  You have to get the height empirically
            % ---------------------------------------------------------%
            % fill each column with content            
            \begin{block}{What was planned to be implemented for the current Iteration?}
              \begin{itemize}
              \item Maintaining user's session history.
              \item Displaying the result of user computation.
              \item Allow User to view their computation history of the current session.
              \item Performing basic arithematic operation i.e. add, multiplication,subtraction,division on two user enetered numbers.
              \item Calculating Natural Logarithm Properties i.e. Quotient Rule,Product Rule,Power Rule on a user enetered number with $ln_{e}2$.
              \item Compute the result of $ln_{e}2$.
              \item Compute the result of natural logarithm of user enetered number.
              \item Compute total time required to double initial priniciple when interest rate is compunded annually.
              \item Compute total time required to double initial priniciple when interest rate is compunded continously.
              \item Compute the inverse funtion of $ln_{e}2$.
              \end{itemize}
            \end{block}
            \vfill
              \begin{block}{What happened in the current Iteration?}
              \begin{itemize}
              \item All planned task completed.
              \item To maintain user history a re-sizeable array ie. arrayList was used.
              \item The calculator built was a text based calculator.
              \end{itemize}              
            
            \end{block}
            \vfill
              \begin{block}{Any critical decisions made?}
              \begin{itemize}
              \item Making a surrogate user for the project to add a famous real world application of $ln_{e}2$ i.e. "Computing half life of a substance" to the Enternity:Numbers which was found by introspection.\\
             \textbf{ Why was this decision critical?}
                \begin{itemize}
                \item It is a very famous problem and commonly used concept in nuclear physics and chemistry.Adding this feature to calculator would be make it more meaningful and relevant to new groups of user community- physicists and chemists.However this feature was not added to the  list of planned task of iteration 1.
                \end{itemize}
              \item Making the calculator a text based calculator and not a UI based for the iteration 1.\\
              \textbf{ Why was this decision critical?}
                \begin{itemize}
                \item The reason to make the calculator a text based calculator was to give a basic workable version of a Enternity:Number in iteration 1 to the user community without getting involved in different technologies and frameworks.The decision was critical as it was tradeoff between expected functionality that the calculator is supposed to offer versus the visual appeal of the calculator. I chose to build a text based calculator. 
                \end{itemize}
                \item Converting a real user to a surrogate user a change that was made in Deliverable 2 from Deliverable1. \\
             \textbf{ Why was this decision critical?}
                \begin{itemize}
                    \item The decision was critical because the real user had no goals and expectation from Enternity:Number and seemed content with the scientific calculators, available in market, for any computation related to $ln_e{2}$. Since the user was a student and was representing the student community as user for Enternity:Number so this user was converted to a surrogate user and relevant goals for instance computing natural logarithm properwere added to this newly created surrogate user.
                \end{itemize}
                    \item Result rounded to 2 or 3 decimal places only for this iteration. \\
             \textbf{ Why was this decision critical?}
                \begin{itemize}
                \item It restricts the user from getting result rounded upto a certain decimal place other than 2,3, 16(default). However I decided to output result of computation available upto 2,3,16 decimal places for iteration 1 because all the users whom I interviewed uses the result either rounded upto 2 decimal place or 3 decimal place.
                \end{itemize}
                      \item Added feature to perform basic arithematic operation on two number, on suggestion of TA.\\
             \textbf{ Why was this decision critical?}
                \begin{itemize}
                \item TA suggested to add a feature to perform arithematic operation (add,subtract,multiple,division) on two user entered numbers.By adding this feature in iteration 1 , these functions can be used as many underlying computational logic for many upcoming features of Enternity:Numbers for example "Computing Basic Arithmetic Operation of a number with lne2".Hence avoiding repeation of code logic. 
                \end{itemize}
              \end{itemize}
              
            
            \end{block}
            \vfill

          }
        \end{minipage}
      \end{beamercolorbox}
    \end{column}
    % ---------------------------------------------------------%
    % end the column

    % ---------------------------------------------------------%
    % Set up a column 
    \begin{column}{.49\textwidth}
      \begin{beamercolorbox}[center,wd=\textwidth]{postercolumn}
        \begin{minipage}[T]{.95\textwidth}  % tweaks the width, makes a new \textwidth
          \parbox[t][\columnheight]{\textwidth}{ % must be some better way to set the the height, width and textwidth simultaneously
            % Since all columns are the same length, it is all nice and tidy.  You have to get the height empirically
            % ---------------------------------------------------------%
            % fill each column with content            
            \begin{block}{What went well in the current Iteration?}
              \begin{itemize}
              \item The result returned by computing the natural logarithm of number was giving a decent precision i.e the result of Enternity:Number matched with result of a scientific calculator upto 4 decimal places.For example:
                \begin{itemize}
                \item Value of $ln_e{2}$ computed by Enternity:Numbers is 0.6931534304818241
                 \item Value of $ln_e{2}$ from scientific calcultaor  Casio model : fx-991MS is 0.69314718
                \end{itemize}
              \item User's session history was maintained for all computations with result being a Math Error or a valid result along with the user's command to Enternity:Numbers.
              \item User entered numbers for a given operation was always validated specially for negative numbers because natural logarithm of a negative number is invalid or Math Error.
                
              \end{itemize}              
            
            \end{block}
            \vfill
              \begin{block}{\smallskip What didnt go well in current Iteration and how can we improve ?}
              \begin{itemize}
              \item Users are not given the flexibility to choose the precision of the result i.e upto what decimal place they would want to get the result to be rounded off.\\
              \textbf{What can be done to improve this ?}
                \begin{itemize}
                \item Take input from the user about the decimal places they would want to  round off the result to and any other input parameters if required by an operation.
                \item Based on the input number of rounding off the result should save and returned upto that decimal place.
                \end{itemize}
              \item Not giving user an option to clear their history of computation.\\
              \textbf{What can be done to improve this ?}
                \begin{itemize}
                \item A new  feature to be added to Enternity Numbers which will make the clearing history possible with a single click of the user.                \end{itemize}
                 \item Converting the text based calculator to UI based calculator.\\
              \textbf{What can be done to improve this ?}
                \begin{itemize}
                \item Use Java JFrames concepts to built a User Interface of Enternity:Number
                \end{itemize}
                  \item Testing was manual in this iteration and should be converted to automated testing.\\
              \textbf{What can be done to improve this ?}
                \begin{itemize}
                \item Use Junit for automated testing.
                \end{itemize}
                    \item All scenario of possible exception were not handled in this iteration  i.e any exception that is thrown in the source should throw a user understandable exception and not Java based exception. \\
              \textbf{What can be done to improve this ?}
                \begin{itemize}
                \item Enhance exception handling ensuring all possible case exception cases are handled. 
            \end{itemize}
              \end{itemize}              
            
            \end{block}
            \vfill
              \begin{block}{Action plan for next iteration}
              Below are top 3 task selected and they will be a part of the next iteration.Remaining Task will be added to Enternity:Numbers as a requirement and will be handled accordingly.These are selected as they bring in more user flexibility and make the Enternity:Numbers more robust.
              \begin{itemize}
              \item  Users are not given the flexibility to choose the precision of the result i.e upto what decimal place they would want to get the result to be rounded off.
              \item Converting the text based calculator to UI based calculator.
              \item Testing was manual in this iteration and should be converted to automated testing.
               \end{itemize}              
            
            \end{block}
            \vfill
              \begin{block}{Lessons Learnt: }
              \begin{itemize}
              \item Benefits of automated testing over manual testing.
              \item More user interviews should be conducted to get more software requirement.
              \end{itemize}              
            
            \end{block}
            \vfill
              \begin{block}{Reference}
              \begin{itemize}
              \item Atlassian. “How to Run an Agile Retrospective Meeting with Examples.” Atlassian, www.atlassian.com/team-playbook/plays/retrospective.
            \item “What Is Agile Retrospective? - Definition from WhatIs.com.” SearchSoftwareQuality, searchsoftwarequality.techtarget.com/definition/Agile-retrospective.
            \item Deselaers. “Deselaers/Latex-Beamerposter.” GitHub, 1 Oct. 2018, github.com/deselaers/latex-beamerposter.
              \end{itemize}              
            
            \end{block}
          }
        \end{minipage}
      \end{beamercolorbox}
    \end{column}
    % ---------------------------------------------------------%
    % end the column
  \end{columns}
  \end{frame}
\end{document}

%%%%%%%%%%%%%%%%%%%%%%%%%%%%%%%%%%%%%%%%%%%%%%%%%%%%%%%%%%%%%%%%%%%%%%%%%%%%%%%%%%%%%%%%%%%%%%%%%%%%
%%% Local Variables: 
%%% mode: latex
%%% TeX-PDF-mode: t
%%% End:
